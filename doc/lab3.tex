\documentclass[a4paper, 12pt]{article}
\usepackage{cmap}
\usepackage[12pt]{extsizes}			
\usepackage{mathtext} 				
\usepackage[T2A]{fontenc}			
\usepackage[utf8]{inputenc}			
\usepackage[english,russian]{babel}
\usepackage{setspace}
\singlespacing
\usepackage{amsmath,amsfonts,amssymb,amsthm,mathtools}
\usepackage{fancyhdr}
\usepackage{soulutf8}
\usepackage{euscript}
\usepackage{mathrsfs}
\usepackage{listings}
\pagestyle{fancy}
\usepackage{indentfirst}
\usepackage[top=10mm]{geometry}
\rhead{}
\lhead{}
\renewcommand{\headrulewidth}{0mm}
\usepackage{tocloft}
\renewcommand{\cftsecleader}{\cftdotfill{\cftdotsep}}
\usepackage[dvipsnames]{xcolor}

\lstdefinestyle{mystyle}{ 
	keywordstyle=\color{OliveGreen},
	numberstyle=\tiny\color{Gray},
	stringstyle=\color{BurntOrange},
	basicstyle=\ttfamily\footnotesize,
	breakatwhitespace=false,         
	breaklines=true,                 
	captionpos=b,                    
	keepspaces=true,                 
	numbers=left,                    
	numbersep=5pt,                  
	showspaces=false,                
	showstringspaces=false,
	showtabs=false,                  
	tabsize=2
}

\lstset{style=mystyle}

\begin{document}
\thispagestyle{empty}	
\begin{center}
	Московский авиационный институт
	
	(Национальный исследовательский университет)
	
	Факультет "Информационные технологии и прикладная математика"
	
	Кафедра "Вычислительная математика и программирование"
	
\end{center}
\vspace{40ex}
\begin{center}
	\textbf{\large{Лабораторная работа №3 по курсу\linebreak \textquotedblleft Операционные системы\textquotedblright}}
\end{center}
\vspace{35ex}
\begin{flushright}
	\textit{Студент: } Живалев Е.А.
	
	\vspace{2ex}
	\textit{Группа: } М8О-206Б
	
	\vspace{2ex}
	\textit{Преподаватель: } Соколов А.А.
	
	\vspace{2ex}
	\textit{Вариант: 6} 
	
	\vspace{2ex}
	\textit{Оценка: } \underline{\quad\quad\quad\quad\quad\quad}
	
	 \vspace{2ex}
	\textit{Дата: } \underline{\quad\quad\quad\quad\quad\quad}
	
	\vspace{2ex}
	\textit{Подпись: } \underline{\quad\quad\quad\quad\quad\quad}
	
\end{flushright}

\vspace{5ex}

\begin{vfill}
	\begin{center}
		Москва, 2019
	\end{center}	
\end{vfill}
\newpage

\section{Задание}

Произвести распараллеленный поиск по ненаправленному графу в ширину. Граф задается
набором значений, что хранятся в вершинах, и набором пар связей. Информация по графу
хранится в отдельном файле. Необходимо определить есть ли в графе циклы.

В ходе выполнения лабораторной работы были использованы следующие системные вызовы:

\begin{itemize}
	\item pthread\_create - создание нового процесса
	\item pthread\_join - ожидание завершения работы потока
	\item pthread\_exit - завершение потока
	\item pthread\_mutex\_lock - захват мьютекса
	\item pthread\_mutex\_unlock - освобождение мьютекса
\end{itemize}
\section{Описание работы программы}

Я попытался реализовать threadpool, в который я мог бы отправлять задачи(обработку конкретной вершины графа) и получать ответ. Я создал две очереди, одна - для отправки задач, вторая - для получения результатов. Поток, обрабатывая вершину графа, отмечает в массиве used, что вершина была посещена и добавляет в очередь заданий все вершины, соединенные с текущей, а также для них в массиве parent отмечает, что текущая вершина является их родителем. Если же вершина, соединенная с текущей, была посещена, то проверяется, что она является родителем текущей вершины, иначе в графе существует цикл. Операции, связанные с добавлением заданий в очереди, а также обновлением массивов used и parent блокируются во избежании гонки за ресурсами.

\newpage

\section{Исходный код}



\textbf{\large{main.c}}
\lstinputlisting[language=C++]{../src/main.cpp}
\newpage
\section{Консоль}

\begin{verbatim}
	qelderdelta@qelderdelta-UX331UA:~/Study/OS/os_lab_3/src/build$ ./lab3
	3 3
	1 2
	2 3
	3 1
	1
	Singlethreaded: Cycle found
	2
	Multithreaded: Cycle found
	qelderdelta@qelderdelta-UX331UA:~/Study/OS/os_lab_3/src/build$ ./lab3
	3 2
	1 2
	2 3
	1
	Singlethreaded: Cycle not found
	2
	Multithreaded: Cycle not found                       
\end{verbatim}
\newpage
\section{Выводы}

В ходе выполнения лабораторной работы я получил очень важный почти для любого разработчика опыт написания многопоточных программ. Конкретно я познакомился с тем, как устроены потоки в ОС Linux, а также с устройством такого примитива синхронизации как мьютекс.
\end{document}